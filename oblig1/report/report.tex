\documentclass[a4paper,11pt]{article}
%\documentclass[preprint]{aa}

%\documentclass[preprint]{aastex}

%\documentclass[journal = ancham]{achemso}
%\setkeys{acs}{useutils = true}
%\usepackage{fullpage}
\usepackage{natbib,twoopt}
\pretolerance=2000
\tolerance=6000
\hbadness=6000
%\usepackage[landscape]{geometry}
%\usepackage{pxfonts}
%\usepackage{cmbright}
%\usepackage[varg]{txfonts}
%\usepackage{mathptmx}
%\usepackage{tgtermes}
\usepackage[utf8]{inputenc}
%\usepackage{fouriernc}
%\usepackage[adobe-utopia]{mathdesign}
\usepackage[T1]{fontenc}
%\usepackage[norsk]{babel}
\usepackage{epsfig}
\usepackage{graphicx}
\usepackage{amsmath}
%\usepackage[version=3]{mhchem}
\usepackage{pstricks}
\usepackage[font=small,labelfont=bf,tableposition=below]{caption}
%\usepackage{subfig}
\usepackage{subcaption}
%\usepackage{varioref}
\usepackage{hyperref}
\usepackage{listings}
\usepackage{sverb}
%\usepackage{microtype}
%\usepackage{enumerate}
\usepackage{enumitem}
%\usepackage{lineno}
%\usepackage{booktabs}
%\usepackage{changepage}
%\usepackage[flushleft]{threeparttable}
\usepackage{pdfpages}
\usepackage{float}
\usepackage{mathtools}
%\usepackage{etoolbox}
%\usepackage{xstring}
\usepackage{aas_macros}

\floatstyle{plaintop}
\restylefloat{table}
%\floatsetup[table]{capposition=top}

\setcounter{secnumdepth}{3}

\newcommand{\tr}{\, \text{tr}\,}
\newcommand{\diff}{\ensuremath{\; \text{d}}}
\newcommand{\diffd}{\ensuremath{\text{d}}}
\newcommand{\sgn}{\ensuremath{\; \text{sgn}}}
\newcommand{\UA}{\ensuremath{_{\uparrow}}}
\newcommand{\RA}{\ensuremath{_{\rightarrow}}}
\newcommand{\QED}{\left\{ \hfill{\textbf{QED}} \right\}}

%% The below macros turn citations into ADS clickers in dvi, pdf, html output.
%% EDP Sciences improved them in December 2012 to work also with pdflatex.
\bibpunct{(}{)}{;}{a}{}{,}    %% natbib cite format used by A&A and ApJ
\makeatletter
 \newcommandtwoopt{\citeads}[3][][]{\href{http://adsabs.harvard.edu/abs/#3}%
   {\def\hyper@linkstart##1##2{}%
    \let\hyper@linkend\@empty\citealp[#1][#2]{#3}}}    %% Rutten, 2000
 \newcommandtwoopt{\citepads}[3][][]{\href{http://adsabs.harvard.edu/abs/#3}%
   {\def\hyper@linkstart##1##2{}%
    \let\hyper@linkend\@empty\citep[#1][#2]{#3}}}      %% (Rutten 2000)
 \newcommandtwoopt{\citetads}[3][][]{\href{http://adsabs.harvard.edu/abs/#3}%
   {\def\hyper@linkstart##1##2{}%
    \let\hyper@linkend\@empty\citet[#1][#2]{#3}}}      %% Rutten (2000)
 \newcommandtwoopt{\citeyearads}[3][][]%
   {\href{http://adsabs.harvard.edu/abs/#3}%
   {\def\hyper@linkstart##1##2{}%
    \let\hyper@linkend\@empty\citeyear[#1][#2]{#3}}}   %% 2000
\makeatother

%\newcommand{\diff}{%
%    \IfEqCase{frac{\diff}{%
%        {\ensuremath{frac{\text{d}} }}%
%        {\ensuremath{\; \text{d}} }% 
%    }[\PackageError{diff}{Problem with diff}{}]%
%}%


\date{\today}
\title{Compulsory assignment spring 2014\\ \small{Statistical mechanics -- FYS4130}}
\author{Marius Berge Eide \\ \texttt{m.b.eide@astro.uio.no}}


\begin{document}


\onecolumn
\maketitle{}


\section{Part 1}

\begin{enumerate}
    \item \textbf{Normalisation and variance}

        The distribution $P_0(x)$ is normalised;
        \begin{align*}
            \int_{-\infty}^{+\infty} P_0(x) \diff x &= \int_{-1}^{+1} \frac{1}{2} \diff x \\
            &= \frac{1}{2} \left[ 1 - \left( -1 \right) \right] = 1
        \end{align*}
        and the variance $\langle \Delta x^2 \rangle$ can be found as $\langle \Delta x^2 \rangle - \langle \Delta x \rangle^2$ with
        \begin{align*}
            \langle x \rangle &= \int_{-\infty}^{+\infty} x P_0(x) \diff x = \int_{-1}^{+1} x \frac{1}{2} \diff x \\
            &= \frac{1}{2} \left[ \frac{1}{2} \left( 1^2 - (-1)^2 \right) \right] = 0
        \end{align*}
        and
        \begin{align*}
            \langle x^2 \rangle &=  \int_{-\infty}^{+\infty} x^2 P_0(x) \diff x = \int_{-1}^{+1} x^2 \frac{1}{2} \diff x \\
            &= \frac{1}{2} \left[ \frac{1}{3} \left( 1^3 - (-1)^3 \right) \right] \\
            &= \frac{1}{6} 2 = \frac{1}{3}
        \end{align*}
        giving
        \begin{align}
            \langle \Delta x^2 \rangle &= \langle \Delta x^2 \rangle - \langle \Delta x \rangle^2 \notag \\
            &= \frac{1}{3} - 0 = \frac{1}{3}.
            \label{eq:variance1}
        \end{align}

    \item \textbf{Random sequence}<++>

    \item \textbf{Histogram}<++>
\end{enumerate}<++>

\section{Part 2 -- The central limit theorem}
The first distribution to be examined is the power law approximation to the log-normal distribution;
\begin{equation}
    P(x) \propto \frac{1}{x}
    \label{eq:powerlawdist}
\end{equation}

\begin{enumerate}
    \item The distribution is not normalisable as
        \begin{equation}
            \int_{-\infty}^{+\infty} \frac{1}{x} \diff x = \infty
            \label{eq:diverging_dist}
        \end{equation}
        alas, the 
\end{enumerate}<++>

%%%%%%%%%%% BIBLIOGRAPHY %%%%%%%%%%%%%%%%%
\bibliography{referanser}
\bibliographystyle{apj}
%\bibliographystyle{astroads}
%\bibliographystyle{apj_hyperref}

\clearpage
\appendix
\section{Appendix}
\label{sec:appendix}

\subsection{Peebles equation}
\label{app:peebles}

%\lstinputlisting[language=c++]{../mainMonteCarloVMC1.cpp}

\end{document}

